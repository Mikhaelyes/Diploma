%класс документа, А4, шрифт
%\documentclass[reprint, amsmath, amssymb, aps,]{revtex4-2}
\documentclass[a4paper,12pt]{article}

\usepackage[T2A]{fontenc} %кодировка
\usepackage[utf8]{inputenc} %кодировка исходного текста
\usepackage[english, russian]{babel} %локализация и переносы

%математика
\usepackage{amsmath, amsfonts, amssymb, amsthm, mathtools}

%графика
\usepackage{graphicx}
\DeclareGraphicsExtensions{.pdf,.png,.jpg}
\usepackage{wrapfig} %обтекание фигур

\usepackage{multirow}
\usepackage{dcolumn}% Align table columns on decimal point
\usepackage{bm}% bold math
\usepackage{caption}
\usepackage{indentfirst}
\usepackage{bbm}

\begin{document}
\title{Тестирование стационарности сообществ в динамических случайных графах}



\author{Васильев Михаил Владимирович}
%\affiliation{%
% Студент 5 курса факультета ФРКТ\\
%}%

%\collaboration{Московский физико-технический институт}%\noaffiliation

\date{11 мая 2024 г.}% It is always \today, today,
             %  but any date may be explicitly specified
             



\maketitle

\section{Введение}

	Цель работы состоит в разработке и сравнении методов тестирования стационарности распределений сообществ и их зависимости при различных методах разбиения на сообщества.

	
	
\section{Задание графа}

\subsection{Задание изначального графа}

	Зададим случайный граф $G_{0}$ в модели Эрдёша - Реньи \cite{book1}.
Пусть дано множество $V_{n} = \left\{ 1,...,n \right\}$, элементы которого мы назовем вершинами. Именно на этом множестве мы будем “строить” случайный граф. Понятно, стало быть, что случайным будет множество ребер графа. Мы не хотим сейчас рассматривать графы с кратными ребрами (мультиграфы), графы с петлями (псевдографы) и ориентированные графы (орграфы). Поэтому мы считаем, что потенциальных ребер у графа не больше, чем $C_{n}^{(2)}$ штук. Будем соединять любые две вершины $i$ и $j$ ребром с некоторой вероятностью $p \in [0, 1]$ независимо от всех остальных $C_{n}^{(2)} - 1$ пар вершин. Иными словами, ребра появляются в соответствии со стандартной схемой Бернулли, в которой $C_{n}^{(2)} - 1$ испытаний и “вероятность успеха” $p$. Обозначим через $E$ случайное множество ребер, которое возникает в результате реализации такой схемы. Положим $G_{0} = (V_{n}, E)$.

\subsection{Эволюция случайного графа}
\subsubsection{Обзор моделей эволюции случайного графа}
	Обозначим граф на шаге эволюции $t$, как $G_{t}=(V_{t},E_{t}),$ $V_{t}$- множество вершин, $E_{t}$- множество ребер. Эволюция, т.е. присоединение новых узлов новыми связями к существующим узлам, начинается с начального графа $G_{0}$. Рост числа узлов и связей в реальных сетях происходит по различным моделям эволюции таким, как предпочтительное присоединение (ПП, preferential attachment), кластерное присоединение (КП, clustering attachment) и их смеси. Смена модели присоединения происходит из-за внешних факторов, например, появление новой области знаний в сети цитирования. Параметры модели могут меняться во времени. Поскольку присоединение новых узлов в моделях ПП и КП имеет вероятностную природу, можно определить $G_{t}$, как динамический случайный граф.
	

\subsubsection{Preferential attachment}
	ПП моделирует присоединение новых узлов с большей вероятностью к наиболее влиятельным узлам, так как вероятность присоединения пропорциональна числу связей узла. Новый узел присоединяется к существующему узлу $i$ ненаправленного графа с вероятностью:

\begin{equation}\label{F1}
P_{PA}(i,t) = \dfrac{d_{i,t}}{\sum\limits_{s \in V_{t}} d_{s,t}},
\end{equation}

где $d_{i,t}$, - число связей узла $i \in V_{t}$ . ПП был предложен для
обоснования тяжелых хвостов распределений характеристик влиятельности узлов таких, как число их связей и Пейджранги, что характерно для Веб-графов.


\subsubsection{Clustering attachment}

	КП предложен в \cite{book6} для моделирования локальных сетей, когда присоединение нового узла происходит лишь к узлам, вовлеченным в треугольники. Например, в социальных сетях индивиды могут иметь друзей не среди популярных людей, а в своем окружении, формируя тесные сообщества. Вместо гигантских узлов с большими числом связей, как для ПП, КП ведет к тесно-связанным сообществам. В показано без доказательства, что КП порождает легкие хвосты распределений числа связей узлов. КП моделирует эволюцию социальных сетей и сетей, где новые узлы присоединяются не к узлам
с большим числом связей, а к сильно связанным сообществам. Вероятность присоединения нового узла к узлу $i \in V_{t} $ определяется как

\begin{equation}\label{F11}
P_{CA}(i,t) = (C_{i,t}^{\alpha} + \varepsilon) / (\sum\limits_{s \in V_{t}} c_{s,t}^{\alpha} + \epsilon \Vert V_{t} \Vert), \quad \alpha > 0,
\end{equation}

\begin{equation}\label{F12}
P_{CA}(i,t) = \dfrac{\mathbbm{1}\left\{c_{i,t}>0 \right\}+\epsilon}{\sum\limits_{j \in V_{t}} \mathbbm{1}\left\{c_{j,t}>0 \right\} + \Vert V_{t} \Vert\epsilon}, \quad \alpha = 0,
\end{equation}

где

\[ c_{i,t} =
  \begin{cases}
    0,       & \quad d_{i,t}=0 \text{ или } d_{i,t}=1,\\
    2 \vartriangle_{i,t} / (d_{i,t}(d_{i,t}-1)),  & \quad d_{i,t} \geq 2,
  \end{cases}
\]

-кластерный коэффициент, $0\leq c_{i,t} \leq1$, $\vartriangle_{i,t}$ число треугольников, в которые вовлечен узел $i$ на шаге эволюции $t$, $\alpha, \epsilon \geq 0$ - параметры модели. Модель КП не позволяет моделировать параллельные связи между парами узлов и петли узлов.

\section{Выявление сообществ в графе}
Сообщество - это группа тесно взаимодействующих узлов, которые слабо связаны с остальным графом.


\section{Стационарность сообществ}
\subsection{Метод случайных последовательностей}
	Одним из подходов для тестирования сообществ на стационарность распределения может быть сведение к тестированию случайных последовательностей, возникающих на случайных графах. для этого могут быть использованы случайные блуждания для сбора информации о характеристиках влиятельности (например число входящих связей узла или PageRank) узлов сообщества, которые определяют соответствующие случайные последовательности.
	



\subsection{Метод разбиения на подграфы}
	Ещё один подход - это разбиение сообщества на более мелкие подграфы, рассматривая их как блоки данных. В каждом блоке можно оценить индекс экстремальной величины или его обратную величину, хвостовой индекс, и понять, насколько сильно меняется его величина. Далее в статье будут описаны методы оценки индекса экстримальной величины.






\section{Методы оценки индекса экстримальной величины}
	Методология оценивания индекса экстримальной величины состоит в следующем: рассматривается стационарный случайный граф G, для каждой его вершины $V_{i}$ рассчитывается значение $X_{i}$, как число связей вершины или её PageRank. Таким образом последовательность значений $X_{i}, 1\leq i \leq n$ пораждает некоторое распределение с тяжёлым или лёгким хвостом. 
 
	Для последовательности сл.в $\left\{ X_{n} \right\}_{n \geq 1}$ обозначим соответствующие порядковые статистики, расположенные в порядке возрастания, как $X_{i,n}$
	
	Оценку ИЭВ $\gamma \in R$ будем производить методом Mixed Moment estimator:
	
		
\begin{equation}\label{F2}
\hat{\gamma}^{MM}(n,k)=\dfrac{\hat{\varphi}_{n}(k)-1}{1+2min(\hat{\varphi}_{n}(k) - 1,0)},
\end{equation}

\begin{equation}\label{F3}
\hat{\varphi}_{n}(k) = \dfrac{M_{n}^{(1)} (k) - L_{n}^{(1)} (k)}{(L_{n}^{(1)} (k))^{2}},
\end{equation}

\begin{equation}\label{F4}
L_{n}^{(1)} (k) = 1 - \dfrac{1}{k} \sum_{i=1}^k \dfrac{X_{n-k,n}}{X_{n-i+1,n}},
\quad
M_{n}^{(1)} (k) = \dfrac{1}{k} \sum_{i=1}^k ln\dfrac{X_{n-i+1,n}}{X_{n-k,n}}.
\end{equation}	
	
	Оценка предназначена для $\gamma \in R$. $\gamma \leq 0$ указывает на лёгкий хвост распределения, а $\gamma > 0$ на тяжёлый по теореме Фишера-Типпета-Гнеденко, \cite{book5}. 
		
\section{Выводы}
	 На данный момент произведено изначальное описание методологии проверки стационарности сообществ.



\newpage
 
% даём указание на включение данного место в оглавление как секции (\section)
\addcontentsline{toc}{section}{Список используемой литературы}

\begin{thebibliography}{0}
\bibitem{book1} Райгородский А. М. Модели случайных графов и их применения // Труды МФТИ. 2010.

\bibitem{book2} Dynamic Networks // Network Repository URL: https://networkrepository.com (дата обращения: 05.05.2024).

\bibitem{book3} Маркович Н.М., Вайсиулюс М.Р. Extreme Value Statistics for Evolving Random Networks // Mathematics. 2023. 11(9). С. 2171 https://www.mdpi.com/2227-7390/11/9/2171.

\bibitem{book4} Nicolas Dugué, Anthony Perez. Directed Louvain : maximizing modularity in directed networks. [Research Report] Université d'Orléans. 2015.

\bibitem{book5} BEIRLANT J., GOEGEBEUR Y., TEUGELS J., SEGERS J. Applications. – Chichester, West Sussex: Wiley, 2004 – 504 p.

\bibitem{book6} BAGROW J., BROCKMANN D. Natural Emergence of Clusters and Bursts in Network Evolution // Physical Review X. – 2012 –V. 3 – No. 2 –P. 21016



\end{thebibliography}



\end{document}