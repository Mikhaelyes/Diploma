%класс документа, А4, шрифт
\documentclass[reprint, amsmath, amssymb, aps,]{revtex4-2}
%\documentclass[a4paper,12pt]{article}

\usepackage[T2A]{fontenc} %кодировка
\usepackage[utf8]{inputenc} %кодировка исходного текста
\usepackage[english, russian]{babel} %локализация и переносы

%математика
\usepackage{amsmath, amsfonts, amssymb, amsthm, mathtools}

%графика
\usepackage{graphicx}
\DeclareGraphicsExtensions{.pdf,.png,.jpg}
\usepackage{wrapfig} %обтекание фигур

\usepackage{multirow}
\usepackage{dcolumn}% Align table columns on decimal point
\usepackage{bm}% bold math
\usepackage{caption}

\begin{document}
\title{Практические задачи по курсу: Методы анализа распределений данных с тяжёлыми хвостами. Вариант: 7}



\author{Васильев Михаил Владимирович}
\affiliation{%
 Студент 5 курса факультета ФРКТ\\
}%

\collaboration{Московский физико-технический институт}%\noaffiliation

\date{16 декабря 2023 г.}% It is always \today, today,
             %  but any date may be explicitly specified
             

%\begin{abstract}
%\textbf{В работе исследуются:} абсолютная активность радиоактивного препарата $Co^{60}$ с использованием каскадного перехода $\gamma$ квантов при его распаде.
%\begin{description}
%\item[Оборудование]
%высоковольтный стабилизированный выпрямитель, сцинтиллятор, кристалл йодистого натрия NaI(Ti), формирователь импульсов, %схема совпадений, пересчётный прибор, образец $Co^{60}$
%\end{description}
%\end{abstract}

\maketitle

\section{Data}
Данные взяты из массив данных моделирования случайных графов, полученных с помощью моделей предпочтительного присоединения и кластерного присоединения с разными параметрами для анализа. Данные представляют из себя стобец Excel размером 500 значений. 

