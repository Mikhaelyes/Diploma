%класс документа, А4, шрифт
%\documentclass[reprint, amsmath, amssymb, aps,]{revtex4-2}
\documentclass[a4paper,12pt]{article}

\usepackage[T2A]{fontenc} %кодировка
\usepackage[utf8]{inputenc} %кодировка исходного текста
\usepackage[english, russian]{babel} %локализация и переносы

%математика
\usepackage{amsmath, amsfonts, amssymb, amsthm, mathtools}

%графика
\usepackage{graphicx}
\DeclareGraphicsExtensions{.pdf,.png,.jpg}
\usepackage{wrapfig} %обтекание фигур

\usepackage{multirow}
\usepackage{dcolumn}% Align table columns on decimal point
\usepackage{bm}% bold math
\usepackage{caption}
\usepackage{indentfirst}
\usepackage{bbm}

\begin{document}
\title{Тестирование стационарности сообществ в динамических реальных графах}



\author{Васильев Михаил Владимирович}
%\affiliation{%
% Студент 5 курса факультета ФРКТ\\
%}%

%\collaboration{Московский физико-технический институт}%\noaffiliation

\date{11 мая 2024 г.}% It is always \today, today,
             %  but any date may be explicitly specified
             



\maketitle

\section{Введение}
	Рассматриваются эволюция реальных ненаправленных сетей и описывающих их графов. 
	Обозначим граф на шаге эволюции $t$, как $G_{t}=(V_{t},E_{t}),$ $V_{t}$- множество вершин, $E_{t}$- множество ребер. Эволюция, т.е. присоединение новых узлов новыми связями к существующим узлам, начинается с начального графа $G_{0}$ и имеет сложную природу. Существуют некотороые методы моделирования эволюции графа такие как предпочтительное присоединение (ПП, preferential attachment), кластерное присоединение (КП, clustering attachment) и их смеси. \cite{book9}.
	
	Изучаются методы разбиения реальных графов на сообщества, такие как алгоритм Leuven \cite{book8}. 
	
	В работе данные взяты из открытого графового репозитория \cite{book2}, а именно динамический граф fb-messages.csv. Граф составлен на основе сообщений студентов Калифорнийского университета в Ирвине в социальной сети, похожей на Facebook. В набор данных входят пользователи, которые отправили или получили хотя бы одно сообщение. Граф имеет три столбца: source, target, time. Количество вершин - 1900, количество рёбер - 61732.	
	
	Цель работы состоит в разработке и сравнении методов тестирования стационарности распределений сообществ при различных методах разбиения на сообщества.



\section{Описательные характеристики}
	Зафиксируем состояние графа на последний момент времени, рассчитаем для каждого его узла PageRank и проведём исследование расспределения значений PageRank. Исследование состоит в описании распределения и оценивании индекса экстримальной величины $\gamma$ разными методами. Методы оценивания включают в себя: оценку Хилла (Hill), оценку отношения (Ratio), оценку моментов (Moment) \cite{book10} и смешанную оценку мементов (Mixed Moment) \cite{book11}. Так же интерес представляет исследование тяжести хвоста распределения.


\section{Тестирование стационарности}
Пусть $(V^{'}_{n}, E^{'}_{n})$ и $(V^{('')}_{n}, E^{('')}_{n})$ два непересекающихся сообщества в графе $(V_{n}, E_{n})$ с соответствующим набором входящих ребер $I^{'}_{n} = \lbrace{I}_{n}(\upsilon), \upsilon \in V^{'}_{n}\rbrace$ и $I^{''}_{n} = \lbrace{I}_{n}(\upsilon), \upsilon \in V^{''}_{n}\rbrace$.
Положим нулевую гипотезу, как
\begin{equation}\label{F1}
H_{0}: \alpha_{1} = \alpha_{2} = \alpha,
\end{equation}

где $\alpha_{1}$ и $\alpha_{2}$ - хвостовые индексы входящих ребер в $I^{'}_{n}$ и $I^{''}_{n}$ соответственно. Модифицированная статистика Филлипса и Лоретана представлено в \cite{book12} и выглядит как:
\begin{equation}\label{F2}
S = \dfrac{k^{*}_{1} (\hat{\alpha}_{2})^{2}(\hat{\alpha}_{1}/\hat{\alpha}_{2}-1)^{2}}{(\hat{\alpha}_{1})^{2} + (k^{*}_1 / k^{*}_2)(\hat{\alpha}_{2})^{2}}
\end{equation}
Где $\hat{\alpha}_{1}$ и $\hat{\alpha}_{2}$ - значения хвостового индекса вычисленные по средствам оценки Хилла с оптимальным выбором наибольших порядковых статистик $k^{*}_{1}$ и $k^{*}_{2}$. 
При нулевой гипотезе статистика S сходится в распределении к случайной величине, имеющей распределение хи-квадрат с одной степенью свободы. Например, при уровне значимости $ 5\% $ областью отклонения будет интервал $[ 3.841, +\infty] $.	

Можно рассмотреть PageRank узлов сообществ для ориентированных графов или степени узлов для неориентированных вместо входящих степеней, а затем проверить нулевую гипотезу для пар сообществ.	
		
\section{Выводы}
	Произведено описание динамического реального графа. Приведены некоторые подходы для проверки стационарности сообществ. В дальнейшем эти подходы будут применены к реальным данным.


\newpage
 
% даём указание на включение данного место в оглавление как секции (\section)
\addcontentsline{toc}{section}{Список используемой литературы}

\begin{thebibliography}{0}
\bibitem{book1} Райгородский А. М. Модели случайных графов и их применения // Труды МФТИ. 2010.

\bibitem{book2} Dynamic Networks // Network Repository URL: https://networkrepository.com (дата обращения: 05.05.2024).

\bibitem{book3} Маркович Н.М., Вайсиулюс М.Р. Extreme Value Statistics for Evolving Random Networks // Mathematics. 2023. 11(9). С. 2171 https://www.mdpi.com/2227-7390/11/9/2171.
FRAGA ALVES M.I., GOMES M.I., DE HAAN L. Mixed moment estimator and location invariant alternatives //Extremes. – 2009 – No. 12 – P. 149-–185.
\bibitem{book4} Nicolas Dugué, Anthony Perez. Directed Louvain : maximizing modularity in directed networks. [Research Report] Université d'Orléans. 2015.

\bibitem{book5} BEIRLANT J., GOEGEBEUR Y., TEUGELS J., SEGERS J. Applications. – Chichester, West Sussex: Wiley, 2004 – 504 p.

\bibitem{book6} BAGROW J., BROCKMANN D. Natural Emergence of Clusters and Bursts in Network Evolution // Physical Review X. – 2012 –V. 3 – No. 2 –P. 21016

\bibitem{book7} Leadbetter, M.R., Lingren, G. Rootz´n, H. (1983). Extremes and Related Properties of Random Sequence and Processes. ch.3, New York: Springer.


\bibitem{book8} Nicolas Dugué, Anthony Perez. Directed Louvain : maximizing modularity in directed networks. [Research Report] Université d'Orléans. 2015.

\bibitem{book9} ARNOLD N.A., MONDRAG Likelihood-based approach to discriminate mixtures of network models that vary in time // Sci. Rep. –2021. –No.11 –P. 5205

\bibitem{book10} DEKKERS A. L. M., EINMAHL J. H. J., DE HAAN L. A Moment Estimator for the Index of an Extreme-Value Distribution // Ann. Statist. –1989. –No. 17 –P. 1833–1855

\bibitem{book11} FRAGA ALVES M.I., GOMES M.I., DE HAAN L. Mixed moment estimator and location invariant alternatives //Extremes. – 2009 – No. 12 – P. 149-–185.

\bibitem{book12} Quintos, C., Fan Z. and P. C. B. Phillips. Structural Change Tests in Tail Behaviour and the Asian Crisis. The Review of Economic Studies 2001, 68, 633–663.


\end{thebibliography}



\end{document}