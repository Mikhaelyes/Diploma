%класс документа, А4, шрифт
\documentclass[reprint, amsmath, amssymb, aps,]{revtex4-2}
%\documentclass[a4paper,12pt]{article}

\usepackage[T2A]{fontenc} %кодировка
\usepackage[utf8]{inputenc} %кодировка исходного текста
\usepackage[english, russian]{babel} %локализация и переносы

%математика
\usepackage{amsmath, amsfonts, amssymb, amsthm, mathtools}

%графика
\usepackage{graphicx}
\DeclareGraphicsExtensions{.pdf,.png,.jpg}
\usepackage{wrapfig} %обтекание фигур

\usepackage{multirow}
\usepackage{dcolumn}% Align table columns on decimal point
\usepackage{bm}% bold math
\usepackage{caption}

\begin{document}
\title{Тестирование стационарности сообществ}



\author{Васильев Михаил Владимирович}
\affiliation{%
 Студент 5 курса факультета ФРКТ\\
}%

\collaboration{Московский физико-технический институт}%\noaffiliation

\date{11 мая 2024 г.}% It is always \today, today,
             %  but any date may be explicitly specified
             



\maketitle

\section{Введение}
Постановка задачи состоит в исследовании стационарности динамического графа во времени. Для этого используется датесет графов из открытых источников таких как [1]. Одним из подходов исследования стационарности сообществ является сведение к тестированию случайных последовательностей, возникающих на графе. Для этого используются случайные блуждания для сбора информации о характеристиках влиятельности (PageRank), узлов сообщества, которые определяют соответствующие случайные последовательности. Далее производится вычисление хвостового индекса на элементах этих последовательностей и посредствам визуального анализа делается вывод о стационарности графа во времени. Ключевым критерием анализа является значительное изменение хвостового индекса между соседними временными промежутками.



\section{Data}
Данные взяты из открытого графового репозитория [1], а именно динамический граф fb-messages.csv. Описание: "Социальная сеть, похожая на Facebook, создана на основе онлайн-сообщества студентов Калифорнийского университета в Ирвине. В набор данных входят пользователи, которые отправили или получили хотя бы одно сообщение." Граф имеет три столбца: source, target, time. Количество вершин - 1900, количество рёбер - 61732.


\section{Описательные характеристики}
Рассмотрим граф как статитеческий на последний момент времени, рассчитаем для каждого его узла PageRank и проведём исследование расспределения. Исследование состоит в оценивании индекса экстримального значения $\gamma$ разными методами и изучении хвоста распределения.






\section{Разбиение на сообщества}






\section{Тестирование стационарности}





\section{Литература}
1) https://networkrepository.com



\end{document}